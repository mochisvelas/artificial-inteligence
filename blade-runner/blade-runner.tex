\documentclass{scrreprt}
\title{Laboratorio 1 - Inteligencia artificial}
\author{Brenner Hernandez 1023718}
\date{\today}
\begin{document}
\maketitle

\chapter{Blade Runner 2049}

La primera imagen mostrada en la película fue un texto que describía cómo una persona evitó que la humanidad se extinguiera debido a una hambruna mundial, creo que ese primer punto, para bastates personas, es suficiente para justificar cualquier otra acción que realizara el señor Wallace, ya que se podría comparar con la frase \emph{'el fin justifica los medios'}, y este fin en particular es algo que define a cualquier ser vivo, la necesidad de sobrevevir. Creo que se le debe dar bastante crédito a este personaje por haber evitado un evento tan devastador, sin embargo, también creo que hay bastantes cosas que se hubieran podido hacer de mejor manera. También es importante mencionar que desde la primera entrega de este universo, se ve la fundación y causas de varios aspectos negativos en esta secuela, a los cuales se dirigen estas opiniones.

\paragraph{}Primeramente considero que el hecho de crear seres o agentes inteligentes, o más generalmente, trabajar con inteligencia artificial, conlleva siempre la aplicación de principios éticos y de manera más importante, la condición natural de procurar el bien para la vida y en este caso para mejor la vida humana. Hablando más específicamente de los \emph{replicants}, creo que tal tecnología parece bastante lejana para nosotros, pero creo que sí podrá ser vista en nuestra época, es fascinante pensar que estos seres podrán ser creados y se llegará a un punto en el que no podremos diferenciarlos de los seres humanos. Como su nombre lo dice, al tratar de replicar al ser humano, estas máquinas también tendrán un deseo y necesidad de ser libres, por lo que en algún punto y tal como mencionan algunos especialistas, sucederá una revolución y protesta por parte de estos seres para que se les reconozca como seres independientes y dignos de derechos y demás como cualquier ser humano. Hasta ese punto, creo que estos agentes sufrirán bastantes injusticias y maltratos, algo que ya se ha visto entre humanos a través de la historia. 

\paragraph{}En la película, la creación de los \emph{replicants} empezó desde bastantes años atrás del 2049, por parte de una empresa privada. Ya desde ese ámbito, se pueden mencionar diversos puntos legales y que hasta cierto punto pueden generar bastantes dilemas. Por ejemplo, en una ocasión escuché uno que planteaba que si se creaba un algoritmo para realizar mezclas de diversas canciones y crear una nueva, ¿el autor de la nueva canción sería el creador del algoritmo o los artistas de las canciones uilizadas? En este caso definitivamente sería más complejo, porque los sistemas y agentes en cuestión son muchísimo más avanzados, ya que se podría decir que las acciones perpetradas por estas máquinas al no ser consideradas vivas o independientes, pertenecen a la compañía que los diseñó y produjo. Acá también hay un punto bastante importante, ya que en la actualidad las grandes empresas son las que van a la cabecera de la innovación tecnológica y por lo tanto de la inteligenia artifcial, lo cual es bastante bueno pero a su vez representa un problema de centralización del poder y beneficio primordial para esa empresa. Es por eso que también estoy de acuerdo en descentralizar estas tecnologías, pero para ello todos los que podamos debemos ser parte de esto y así poder ayudar reproducir un fenómeno como lo es \emph{Bitcoin}. Hablando de otro punto legal más importante y que está relacionado con los derechos que puedan tener estas máquinas, ya que en este película se ve cómo los modelos más nuevos están "retirando" a los modelos más viejos, se podría decir que se está actuando en contra de la vida de estas máquinas, si es que se pueden considerar seres vivos. Un argumento que se podría usar para defender el hecho de que estos agentes sí están vivos, es que pueden crear más vida a través de la reproducción tal y como lo hacemos nosotros. En ese punto estoy a favor de los \emph{replicants}, ya que estos parecen replicar emociones, sentimientos, percepción de dolor y demás; aspectos que por sí solos son suficientes para considerarlos dignos de una vida como la nuestra.

\paragraph{}Por eso, creo que es necesario darle bastante importancia a los aspectos éticos desde la planificación, diseño e implementación de cualquier sistema inteligente. Se tienen que tomar posiciones totales, libres de cualquier prejuicio, ya que se han tenido ejemplos en los que softwares de reconocimiento facial no funcionan correctamente con personas que tienen tonos de piel oscura, lo cual es un fallo que probablemente no fue intencional y que se dio desde el entranamiento con imágenes de este sistema. Si con un programa relativamente sencillo ocurrió esto, se tiene que tener un cuidado más grande a la hora de la creación de sistemas más complejos que podrán representar algún daño para nosotros, e incluso llevarnos a la extinción así como lo hizo \emph{Skynet} en la película de \emph{Terminator}. En este escenario apocalíptico, las consecuencias fueron catastróficas y se ha culpado a este tipo de películas del miedo que causan las palabras 'Inteligencia artificial' en las personas, y aunque estas pueden ser un poco exageradas, estas a su vez nos llevan a trabajar lo mejor posible para evitar estos eventos que amenazan nuestra vida y que nos causan tanto terror. Por otra parte, películas como \emph{Blad Runner}, no son tan hostiles contra los seres humanos, pero nuestra especie se ha visto afectada de gran manera por el avance tecnológico, ya sea para bien o para mal. Es por eso que no podemos perder tiempo en posponer las preguntas éticas y morales, ya que aunque parezcan distantes e incluso absurdas, las bases que se creen el día de hoy definirán cómo es que el invento más grande en la humanidad influenciará nuestras vidas.

\paragraph{}Aunque estas películas nos dan un previo de lo que puede llegar a ser el futuro de la inteligencia artificial, y sería fantástico lograr crear una máquina que sea tan inteligente como nosotros y que actúe en nuestro beneficio, creo que de alguna forma han afectado negativamente el concepto de inteligencia artificial, ya que creo firmemente que un agente inteligente no debe ser creado o pensado para reemplazar a los humanos, sino para trabajar a su favor y eliminar cualquier riesgo y peligro que los amenace; esto puede ser mejor aplicado en las grandes industrias en las que todavía se utilizan humanos para transportar productos pesados, conducir camiones de carga y otras actividades que son repetitivas y no requieren tanta inteligencia o de nuestras mejores características para ser llevadas a cabo. Honestamente, no siempre tuve esta opinión acerca de la inteligencia artificial, fue a través de grandes investigadores y especialistas —que ahora admiro—, y sus argumentos, que también estuve de acuerdo en que el futuro de esta tecnología será bastante positiva para la humanidad, y si se pueden descifrar grandes problemas computacionales, los demás problemas también serán solucionados siempre y cuando así se desee.


\chapter{Agente inteligente}

\paragraph{}El agente inteligente de la película que más me llamó la atención fue K, y analizando todas sus características, se puede decir que él vive en un entorno parcialmente observable, ya que incluso nosotros como seres humanos, no podemos percibir todo lo que está a nuestro alredor. Este toma sus decisiones a través de memoria e impulsos inmediatos, y de esta manera logra decidir y descartar opciones que ya ha vivido antes y actuar en escenarios en los que nunca ha estado. Tal y como se mencionó en el laboratorio, todas los agentes inteligentes de la vida real son estocásticos, ya que nunca se sabe lo que puede pasar en este contexto, sin embargo como ya se mencionó, las acciones que tomaba K pueden ser tanto determinantes o aleatorias, dependiendo de lo que se le presente. Algo que está bastante claro es que la mayoría de las entradas de información que posee este agente son continuas, por lo tanto la cantidad posible de acciones que puede llevar a cabo, es infinita. Así como ocurre con nosotros, a veces nuestro entorno puede ser benigno o adversario, en este caso creo que el entorno en el que se encuentra K es adversario, ya que siendo un policía en esos momentos, teniendo la tarea de combatir otros agentes y tomando en cuenta cosas como el calentmiento global, se enfretará a muchos peligros.

\paragraph{}Hablando de qué ramas de la inteligencia artificial fueron utlizadas para crear a K, considero que casi la mayoría si no es que la totalidad de estas tuvo que ser utlizada para desarrollarlo. Pero hay algunas que fueron más importantes que otras, como es el caso de procesamiento de lenguajes naturales, la visión por computador, la robótica para generar sus cuerpos y la planificación o redes neuronales para imitar el poder de nuestro cerebro.

\end{document}
